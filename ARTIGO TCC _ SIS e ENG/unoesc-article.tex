\documentclass[
  article,
  a4paper,
  12pt,
  fleqn,
  oneside,
  chapter = TITLE,
  section = TITLE,
]{abntex2}

\usepackage[
  BibURLs = false,
  ABNTNum = none,
]{unoesc-article}

\usepackage{caption}


\addbibresource{unoesc-article.bib}

\titulo{AgroBot: Uso de Chatbot no WhatsApp para Conectar Produtores e Compradores de Insumos Agrícolas}

\autor{%
  Pedro Augusto Larentis Maldaner%
  \thanks{%
    \affil{Acadêmico de Sistemas de Informação; UNOESC; Chapecó}%
    \sep\email{pedro.augusto@unoesc.edu.br}%
  }%
  \and Pedro Henrique Tormem%
  \thanks{%
    \affil{Acadêmico de Sistemas de Informação; UNOESC; Chapecó}%
    \sep\email{pedro.tormem@unoesc.edu.br}%
  }%
  \and William Breno Antunes de Lima de Oliveira%
  \thanks{%
    \affil{Acadêmico de Sistemas de Informação; UNOESC; Chapecó}%
    \sep\email{william.o@unoesc.edu.br}%
  }%
  \and Prof. Jacson Luiz Matte% 
  \thanks{%
    \affil{Especialista em Desenvolvimento de Aplicações Web; UNOPAR; Chapecó}%
    \sep\email{jacson.matte@unoesc.edu.br}%
  }%
}

\data{}

\makeindex
\crefname{figure}{Figura}{Figuras}
\crefname{table}{Quadro}{Quadros}

\renewcommand{\thefootnote}{} 
\makeatletter
\let\@fnsymbol\@arabic
\makeatother

\begin{document}

\pretextual

\begin{paginadetitulo}


\end{paginadetitulo}

\textual


\section{Introdução}\label{sec:intro}

O setor agrícola enfrenta desafios constantes na aproximação entre produtores, fornecedores de insumos e compradores. Muitas vezes, a comunicação é fragmentada, feita por canais informais e sem organização centralizada, o que dificulta o acesso a sementes, insumos e até mesmo ao aluguel de terras.
Diante da popularização das tecnologias de mensagens instantâneas, como o WhatsApp, surge a oportunidade de utilizar chatbots como ferramenta prática e acessível para facilitar essa interação.
Este projeto propõe o desenvolvimento do AgroBot, um chatbot no WhatsApp que funcione como canal automatizado de comunicação entre agricultores, permitindo que produtores publiquem e acessem anúncios de insumos, sementes, maquinários e até áreas para arrendamento. O sistema oferecerá informações rápidas, organização das demandas e maior visibilidade às oportunidades de negociação, fortalecendo as trocas diretas entre agricultores. Além disso, contará com um algoritmo de recomendação inteligente, que sugere produtos e serviços de acordo com os interesses do usuário, seu histórico de interações e a sazonalidade das culturas.


\section{Delimitação do Tema}\label{ssec:teor}

O projeto está delimitado ao desenvolvimento de um chatbot para WhatsApp, que servirá como plataforma de intermediação entre agricultores. O foco é facilitar o contato direto entre produtores rurais, eliminando intermediários e permitindo que agricultores anunciem e encontrem produtos, serviços e equipamentos. O escopo contempla funcionalidades como:

\begin{itemize}
  \item •	Cadastro de agricultores via conversa automatizada;
  \item •	Publicação e listagem de anúncios de insumos, sementes, áreas para arrendamento e maquinários;
  \item Busca inteligente por categorias (sementes, fertilizantes, terras, maquinários etc.);
  \item Sistema de recomendação inteligente, capaz de sugerir produtos e serviços com base nos interesses do usuário, seu histórico de buscas e a sazonalidade agrícola;
  \item Canal de comunicação rápido para negociações;
  \item Possibilidade futura de integração com sistemas de pagamento ou logística.
\end{itemize}

\section{Justificativa}

O uso de aplicativos de mensagens já é consolidado no Brasil, sendo o WhatsApp a principal ferramenta de comunicação entre agricultores e comunidades rurais. Desenvolver um aplicativo próprio exigiria mais tempo e custos, além da necessidade de convencer usuários a instalarem algo novo.
O chatbot, por outro lado, aproveita uma plataforma já difundida, reduzindo barreiras de uso e aumentando as chances de adoção. Essa solução permite democratizar o acesso à informação, reduzir intermediários e agilizar negociações agrícolas, fortalecendo o setor com uma tecnologia simples e de alto impacto.
Além disso, a inclusão de anúncios de insumos, sementes, áreas para arrendamento e maquinários entre agricultores amplia ainda mais o impacto da solução, pois cria uma rede colaborativa de troca e negociação, que favorece especialmente os pequenos e médios produtores. Dessa forma, o AgroBot não apenas aproxima agricultores, mas também promove economia solidária e acesso a recursos que muitas vezes ficam concentrados em grandes mercados.



\section{Objetivo Geral}

Desenvolver um chatbot no WhatsApp para aproximar produtores e compradores agrícolas, facilitando a negociação de insumos, sementes e áreas para arrendamento, de forma prática, acessível e organizada.
Objetivos Específicos:

\begin{itemize}
  \item Implementar um sistema de cadastro automatizado de produtores e compradores;
  \item Disponibilizar listagem e divulgação de produtos, serviços, áreas agrícolas e maquinários;
  \item Criar uma busca inteligente por categorias;
  \item Proporcionar um canal de comunicação ágil e direto entre agricultores;
  \item Desenvolver um sistema de recomendação personalizada, que sugira produtos e serviços de acordo com o perfil do usuário, seu histórico de interações e a sazonalidade agrícola.
\end{itemize}

\end{document}
