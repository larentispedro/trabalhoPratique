\documentclass[
  article,
  a4paper,
  12pt,
  fleqn,
  oneside,
  chapter = TITLE,
  section = TITLE,
]{abntex2}

\usepackage[
  BibURLs = false,
  ABNTNum = none,
]{unoesc-article}

\usepackage{caption}
\usepackage{graphicx}
\usepackage{float}

\setlength{\headheight}{16pt}

\addbibresource{referencias.bib}

\titulo{AgroBot: Uso de Chatbot no WhatsApp para Conectar Produtores e Compradores de Insumos Agrícolas}

\autor{%
  Pedro Augusto Larentis Maldaner%
  \thanks{%
    \affil{Acadêmico de Sistemas de Informação; UNOESC; Chapecó}%
    \sep\email{pedro.augusto@unoesc.edu.br}%
  }%
  \and Pedro Henrique Tormem%
  \thanks{%
    \affil{Acadêmico de Sistemas de Informação; UNOESC; Chapecó}%
    \sep\email{pedro.tormem@unoesc.edu.br}%
  }%
  \and William Breno Antunes de Lima de Oliveira%
  \thanks{%
    \affil{Acadêmico de Sistemas de Informação; UNOESC; Chapecó}%
    \sep\email{william.o@unoesc.edu.br}%
  }%  
  \and Prof. Jacson Luiz Matte% 
  \thanks{%
    \affil{Especialista em Desenvolvimento de Aplicações Web; UNOPAR; Chapecó}%
    \sep\email{jacson.matte@unoesc.edu.br}%
  }%
}

\data{}

\makeindex
\crefname{figure}{Figura}{Figuras}
\crefname{table}{Quadro}{Quadros}

% Remove símbolos (*, †, ‡, §) dos autores
\renewcommand{\thefootnote}{} 
\makeatletter
\let\@fnsymbol\@arabic
\makeatother

\begin{document}

\pretextual

% ========== CAPA ==========
\begin{paginadetitulo}

 \begin{ambienteresumo}%% Resumo
O AgroBot é um chatbot desenvolvido para WhatsApp com o objetivo de facilitar a comunicação e a negociação entre produtores e compradores de insumos agrícolas. A solução busca suprir a dificuldade enfrentada por agricultores — especialmente pequenos e médios — na divulgação de produtos, troca de informações e acesso a oportunidades de compra e venda. Utilizando uma plataforma já familiar e amplamente utilizada no meio rural, o projeto elimina a necessidade de instalar novos aplicativos e oferece uma interface simples, direta e acessível.

O sistema permitirá o cadastro automático de usuários, publicação e consulta de anúncios de insumos, sementes, maquinários e áreas para arrendamento. Além disso, oferecerá filtragem por categorias, contato direto entre produtores e notificações automáticas de novas ofertas. O AgroBot propõe-se a preencher lacunas existentes em soluções atuais, combinando a praticidade do WhatsApp com funcionalidades típicas de um marketplace agrícola, oferecendo uma experiência inclusiva, moderna e eficiente para o agronegócio.
   
    \end{ambienteresumo}

\end{paginadetitulo}

% ======= FORÇA O TEXTO A COMEÇAR NA SEGUNDA PÁGINA =======
\newpage
\textual

% ========== CONTEÚDO ==========
\section{Introdução}\label{sec:intro}

O setor agrícola enfrenta desafios constantes na aproximação entre produtores, fornecedores de insumos e compradores. Muitas vezes, a comunicação é fragmentada, feita por canais informais e sem organização centralizada, o que dificulta o acesso a sementes, insumos e até mesmo ao aluguel de terras.
Diante da popularização das tecnologias de mensagens instantâneas, como o WhatsApp, surge a oportunidade de utilizar chatbots como ferramenta prática e acessível para facilitar essa interação.
Este projeto propõe o desenvolvimento do AgroBot, um chatbot no WhatsApp que funcione como canal automatizado de comunicação entre agricultores, permitindo que produtores publiquem e acessem anúncios de insumos, sementes, maquinários e até áreas para arrendamento. O sistema oferecerá informações rápidas, organização das demandas e maior visibilidade às oportunidades de negociação, fortalecendo as trocas diretas entre agricultores. Além disso, no futuro poderá contar com um algoritmo de recomendação inteligente, que sugere produtos e serviços de acordo com os interesses do usuário, seu histórico de interações e a sazonalidade das culturas.


\section{Delimitação do Tema}\label{ssec:teor}

O trabalho está delimitado ao desenvolvimento de um chatbot agrícola que opera dentro do WhatsApp, permitindo que produtores publiquem anúncios e compradores visualizem essas ofertas de maneira simples e acessível. O sistema realiza apenas cadastro básico, exibição de anúncios e redirecionamento do comprador para o WhatsApp do vendedor, sem realizar negociações, pagamentos ou qualquer forma de intermediação automatizada.

Para sua implementação, o projeto utiliza a WhatsApp Business API, que possibilita a automatização das conversas, integração com banco de dados e criação dos fluxos de interação. No entanto, os usuários finais não precisam da API e podem utilizar o chatbot em qualquer versão do WhatsApp, seja o aplicativo comum ou o WhatsApp Business.

O escopo inclui menus específicos conforme o perfil do usuário (comprador, vendedor ou ambos) e funcionalidades simples de gestão de anúncios, mantendo a solução objetiva, organizada e com foco em facilitar a conexão direta entre agricultores.

\section{Justificativa}

O uso de aplicativos de mensagens já é consolidado no Brasil, sendo o WhatsApp a principal ferramenta de comunicação entre agricultores e comunidades rurais. Desenvolver um aplicativo próprio exigiria mais tempo e custos, além da necessidade de convencer usuários a instalarem algo novo.
O chatbot, por outro lado, aproveita uma plataforma já difundida, reduzindo barreiras de uso e aumentando as chances de adoção. Essa solução permite democratizar o acesso à informação, reduzir intermediários e agilizar negociações agrícolas, fortalecendo o setor com uma tecnologia simples e de alto impacto.
Além disso, a inclusão de anúncios de insumos, sementes, áreas para arrendamento e maquinários entre agricultores amplia ainda mais o impacto da solução, pois cria uma rede colaborativa de troca e negociação, que favorece especialmente os pequenos e médios produtores. Dessa forma, o AgroBot não apenas aproxima agricultores, mas também promove economia solidária e acesso a recursos que muitas vezes ficam concentrados em grandes mercados.



\section{Objetivo Geral}

Desenvolver um chatbot no WhatsApp para aproximar produtores e compradores agrícolas, facilitando a negociação de insumos, sementes e áreas para arrendamento, de forma prática, acessível e organizada.

Objetivos Específicos:

\begin{itemize}
  \item Implementar um sistema de cadastro automatizado de produtores e compradores;
  \item Disponibilizar listagem e divulgação de produtos, serviços, áreas agrícolas e maquinários;
  \item Criar uma busca inteligente por categorias;
  \item Proporcionar um canal de comunicação ágil e direto entre agricultores;
  \item Desenvolver um sistema de recomendação personalizada, que sugira produtos e serviços de acordo com o perfil do usuário, seu histórico de interações e a sazonalidade agrícola.
\end{itemize}

\section{Trabalhos Relacionados}

Durante a pesquisa para o desenvolvimento do AgroBot, constatou-se que \textbf{não existem sistemas que combinem as mesmas características propostas pelo projeto}: um chatbot integrado ao WhatsApp cujo objetivo principal é facilitar a compra e venda direta entre agricultores por meio de anúncios regionais e contato imediato com o vendedor. Assim, os trabalhos relacionados apresentados nesta seção não representam concorrentes diretos, mas sim \textbf{sistemas parcialmente semelhantes}, escolhidos por possuírem elementos tecnológicos, setoriais ou funcionais que se aproximam ainda que de forma limitada da proposta do AgroBot.

A análise crítica desses sistemas permite identificar boas práticas, limitações e, principalmente, lacunas de mercado que justificam a relevância e a originalidade da solução desenvolvida. As reflexões aqui apresentadas também se apoiam nos artefatos coletados com agricultores (entrevistas, mensagens e dados dos formulários), discutidos nas Seções 8 e 9.

\subsection{Nestlé — Theo}

O ``Theo'' é um chatbot criado pela Nestlé para auxiliar produtores de cacau \cite{nestle_theo_mpm}, oferecendo conteúdos técnicos sobre manejo agrícola e boas práticas, diretamente no WhatsApp. Seu propósito é educacional, focado na transmissão de orientações precisas utilizando linguagem simples e acessível.

\textbf{Pontos fortes:} demonstra a viabilidade do uso de chatbots na agricultura via WhatsApp; reforça que agricultores aceitam bem essa tecnologia; utiliza linguagem adequada ao público rural.

\textbf{Limitações:} atende apenas uma cultura específica (cacau); não funciona como marketplace; não permite publicação de anúncios; não conecta compradores e vendedores; não promove interação comercial entre produtores.

\textbf{Discussão crítica:} O Theo mostra que agricultores respondem positivamente a chatbots, mas também evidencia uma limitação clara do mercado: \textbf{não existem chatbots agrícolas voltados para compra e venda}. As entrevistas realizadas reforçam que os agricultores valorizam sistemas simples e rápidos, o que o Theo oferece, mas não resolve suas dores comerciais mais urgentes. Assim, o AgroBot se diferencia ao atuar exatamente onde o Theo não atua.

\subsection{Agrofy}

O Agrofy é o maior marketplace agrícola da América Latina \cite{agrofy}, oferecendo uma plataforma robusta de compra e venda de insumos, sementes, máquinas e serviços. Seu funcionamento é similar a grandes e-commerces tradicionais, com filtros, categorias, anúncios profissionais e alto volume de ofertas.

\textbf{Pontos fortes:} variedade de produtos; estrutura consolidada; presença de fornecedores grandes e pequenos; ampla rede de anunciantes.

\textbf{Limitações:} não utiliza WhatsApp como interface principal; depende de navegadores e cadastros complexos; exige maior familiaridade com tecnologia; não é regionalizado de forma simples; não funciona de forma conversacional.

\textbf{Discussão crítica:} Os formulários aplicados mostram que 100\% dos agricultores utilizam diariamente o WhatsApp, e a maioria afirma preferir soluções simples. A complexidade do Agrofy o torna pouco acessível para muitos pequenos produtores, que relataram dificuldades em navegar por plataformas grandes. O AgroBot, por ser totalmente conversacional, se diferencia ao reduzir barreiras de entrada tecnológicas e ao aproximar produtores locais entre si, algo que o Agrofy não consegue fazer com a mesma proximidade.

\subsection{Agritek}

A Agritek desenvolve chatbots e automações voltadas para empresas do setor agrícola \cite{agritek}, especialmente revendas e consultorias. O seu bot é utilizado para atendimento automatizado, envio de catálogos e suporte técnico.

\textbf{Pontos fortes:} uso real de automações via WhatsApp; atendimento instantâneo; comunicação eficiente entre empresa e produtor.

\textbf{Limitações:} o foco é corporativo; não é um ambiente aberto; não permite que o agricultor publique seus próprios anúncios; não conecta produtor a produtor; não é um marketplace.

\textbf{Discussão crítica:} Este é o trabalho relacionado mais próximo tecnicamente do AgroBot, pois também faz uso do WhatsApp Business API. No entanto, enquanto o Agritek Bot funciona como um canal comercial unidirecional — empresa $\rightarrow$ produtor — o AgroBot opera no modelo \textbf{produtor $\leftrightarrow$ produtor}, que foi exatamente a necessidade mais mencionada pelos entrevistados. Assim, mesmo que exista similaridade tecnológica, a finalidade é completamente diferente.

\subsection{WhatsWave Agro}

O WhatsWave Agro é uma plataforma de automação para o setor rural \cite{whatswave}, geralmente utilizada para envio de alertas, comunicação interna e integração com sensores, máquinas ou sistemas de gestão rural.

\textbf{Pontos fortes:} alta capacidade de automação; integração com infraestrutura agrícola; eficiência administrativa.

\textbf{Limitações:} não atua como marketplace; não exibe anúncios; não facilita negociação; não atende pequenos produtores; não coleta dados simples de uso.

\textbf{Discussão crítica:} Esta solução confirma que o WhatsApp é um meio poderoso para o setor agrícola, mas também mostra que as ferramentas existentes se concentram em gestão e não em comércio. Os agricultores entrevistados demonstraram pouco interesse em automações avançadas, reforçando que o AgroBot atende a uma demanda distinta e mais emergente: \textbf{a facilidade em encontrar compradores e vendedores locais rapidamente}.

\subsection{Análise Comparativa e Lacunas Identificadas}

A comparação entre os sistemas estudados permite identificar um padrão claro: todos eles resolvem \textbf{partes} do problema agrícola, mas nenhum resolve a \textbf{conexão comercial direta} entre produtores de forma simples, acessível e centrada no WhatsApp.

\begin{itemize}
    \item O Theo é um chatbot, mas não é marketplace.  
    \item O Agrofy é marketplace, mas não é chatbot nem funciona via WhatsApp.  
    \item O Agritek Bot usa WhatsApp, mas não conecta produtores entre si.  
    \item O WhatsWave Agro automatiza rotinas, mas não realiza intermediação comercial.  
\end{itemize}

Os artefatos coletados com agricultores reforçam que há demanda por uma ferramenta:

\begin{itemize}
    \item simples e direta;  
    \item baseada no WhatsApp;  
    \item sem necessidade de instalar novos aplicativos;  
    \item regionalizada;  
    \item que facilite a compra e venda rápida.  
\end{itemize}

Estas necessidades, confirmadas nas Seções 8 e 9, não são atendidas por nenhuma solução existente.

\subsection{Diferencial e Originalidade do AgroBot}

O estudo dos trabalhos relacionados demonstra que o AgroBot ocupa uma lacuna importante no setor. A solução une elementos que nunca foram combinados em um único sistema:

\begin{itemize}
    \item chatbot totalmente conversacional;  
    \item integrado ao WhatsApp;  
    \item voltado para compra e venda regional;  
    \item sem intermediários;  
    \item focado em pequenos produtores;  
    \item com menus personalizados e fluxo simplificado.  
\end{itemize}

Essa combinação torna o AgroBot uma proposta inédita e exatamente alinhada com as necessidades e limitações observadas na comunidade agrícola.



\section{Requisitos Funcionais}
\begin{itemize}

  \item \textbf{RF01 – Cadastro via conversa:} o bot coleta nome, município/estado e cultura principal durante uma conversa guiada.
  \item \textbf{RF02 – Publicar anúncio:} fluxo guiado onde o bot solicita categoria, descrição e local do anúncio.
  \item \textbf{RF03 – Listar anúncios:} permite ao usuário buscar por termos ou categorias e visualizar resultados paginados.
  \item \textbf{RF04 – Filtro simples por categoria e local:} busca refinada por tipo e região.
  \item \textbf{RF05 – Contato entre comprador e vendedor:} exibe o número do anunciante ou inicia mensagem automática.
  \item \textbf{RF06 – Gerenciar anúncios:} comandos para editar, renovar ou excluir.
  \item \textbf{RF07 – Notificações proativas:} envio de mensagens com novas ofertas regionais.

\end{itemize}


\section{Requisitos Não Funcionais}
\begin{itemize}
  \item \textbf{RNF01 – Interface simples e intuitiva:} linguagem acessível e menus guiados, adequados a produtores com pouca experiência tecnológica.
  \item \textbf{RNF02 – Compatibilidade com dispositivos móveis:} integração via WhatsApp Business API, sem necessidade de instalar outros aplicativos.
  \item \textbf{RNF03 – Tempo de resposta e desempenho:} respostas automáticas com média inferior a 3 segundos.
  \item \textbf{RNF04 – Segurança e privacidade:} proteção dos dados pessoais e uso ético das informações.
  \item \textbf{RNF05 – Disponibilidade e manutenção:} arquitetura leve e de fácil manutenção.
  \item \textbf{RNF06 – Disponibilidade:} Código aberto e acessível no GitHub
\end{itemize}


\section{Artefatos de Pesquisa e Informações Coletadas na Comunidade}

Durante o desenvolvimento das etapas 1 e 2, foram coletadas informações junto a agricultores, usuários potenciais e membros da comunidade, com o objetivo de compreender a realidade do público-alvo e validar as funcionalidades propostas para o AgroBot. Os dados foram reunidos por meio de conversas informais, mensagens de WhatsApp, anotações de campo e observações sobre o comportamento dos usuários em relação a anúncios agrícolas.

As principais descobertas desta etapa incluem:

\begin{itemize}
    \item Preferência pelo uso do WhatsApp como canal único de comunicação, reforçando a necessidade de evitar a criação de um aplicativo novo.
    \item Dificuldade de pequenos agricultores em encontrar compradores para sementes, insumos e pequenos lotes de maquinário.
    \item Falta de centralização de anúncios agrícolas, fazendo com que agricultores dependam de grupos informais, onde informações se perdem rapidamente.
    \item Interesse em uma solução simples, sem telas complexas ou cadastros extensos.
    \item Importância de evitar intermediários, pois os agricultores preferem negociar diretamente com compradores.
    \item Demanda por anúncios organizados por categorias e região, facilitando a busca por produtos agrícolas.
\end{itemize}

Além disso, foram coletados prints, e-mails, conversas e demais evidências, que estão armazenados e organizados em um repositório público no GitHub.

\begin{center}
\textbf{Link do repositório com os artefatos:} \\
\url{https://github.com/larentispedro/trabalhoPratique}
\end{center}

Esses materiais embasaram a definição dos requisitos funcionais e não funcionais, assim como aprimoraram a compreensão do problema real enfrentado pela comunidade agrícola. As informações coletadas reforçam que o AgroBot atende a uma demanda concreta e recorrente entre agricultores, consolidando sua importância como solução acessível e alinhada ao cotidiano rural.

\section{Análise e Discussão dos Artefatos Coletados}

Além da coleta de evidências via conversas e observações diretas, foram aplicados dois formulários que totalizaram 30 respostas. Também foram reunidos prints de plataformas concorrentes (Agrofy e Agritek) e entrevistas com agricultores, todos disponíveis no repositório GitHub do projeto.

\subsection{Análise dos Dados Coletados nos Formulários}

Os formulários revelaram informações essenciais sobre o comportamento, preferências e dificuldades enfrentadas pelos agricultores. Os dados estão representados graficamente no documento ``Gráficos do formulário coletado'' disponível no GitHub.

Os principais resultados são:

\begin{itemize}
    \item \textbf{Uso do WhatsApp}: 100\% dos participantes utilizam o WhatsApp diariamente. Isso confirma que a escolha da plataforma é adequada para o público-alvo.
    \item \textbf{Interesse em anúncios agrícolas}: 87,5\% afirmaram que gostariam de receber ofertas de maquinário agrícola e insumos diretamente no WhatsApp.
    \item \textbf{Comparação de preços}: 71,4\% utilizariam uma plataforma que mostrasse preços de diferentes fornecedores.
    \item \textbf{Alta prioridade para consulta de insumos}: 78,6\% consideram essa funcionalidade a mais urgente.
    \item \textbf{Problemas reais no plantio}: 53,8\% já enfrentaram atrasos devido a dificuldades na compra de insumos, sementes ou arrendamento de terras.
\end{itemize}

Esses dados reforçam que o AgroBot deve priorizar funcionalidades relacionadas à listagem de insumos, preços, anúncios regionais e simplicidade no acesso.

\subsection{Síntese das Entrevistas com Agricultores}

As entrevistas realizadas via WhatsApp forneceram insights qualitativos importantes. Os participantes reforçaram:

\begin{itemize}
    \item A necessidade de uma solução prática e rápida.
    \item O desejo de evitar intermediários nos processos de compra e venda.
    \item A dificuldade em encontrar fornecedores ou compradores confiáveis.
    \item A validação direta da ideia: agricultores classificaram o projeto como “inovador”, “importante” e “muito útil”.
    \item Sugestões de uso futuro de IA para personalização de ofertas.
\end{itemize}

Os relatos confirmam que o sistema deve ser simples, direto e integrado ao WhatsApp, evitando funcionalidades complexas ou cadastros longos.

\subsection{Impacto dos Artefatos nos Requisitos do Sistema}

Com base nos artefatos coletados, foi possível:

\begin{itemize}
    \item \textbf{Confirmar} requisitos já existentes, como cadastro simples, exibição de anúncios e contato direto com o vendedor.
    \item \textbf{Justificar} as prioridades do sistema: lista de insumos e ofertas regionais são mais importantes do que funcionalidades avançadas no primeiro momento.
    \item \textbf{Identificar novas necessidades}, como filtros regionais, categorias específicas e possibilidade de sugestões personalizadas.
    \item \textbf{Validar o uso exclusivo do WhatsApp}, já que a adoção da plataforma é universal entre os entrevistados.
\end{itemize}

Dessa forma, os artefatos coletados não apenas validam a proposta do AgroBot, mas também orientam sua evolução, garantindo que o sistema permaneça alinhado às necessidades reais da comunidade rural.

\section{Modelagem do Sistema}

A modelagem do sistema tem como objetivo representar, de forma visual e organizada, o funcionamento do AgroBot a partir dos requisitos levantados nas etapas anteriores. Como o sistema proposto trata-se de um chatbot baseado no WhatsApp, a modelagem utiliza diagramas UML que descrevem as principais interações entre o usuário e o bot, bem como os fluxos internos de execução das funcionalidades.

Conforme orientação do professor, foram desenvolvidos os seguintes diagramas UML: 
(i) Diagrama de Casos de Uso, 
(ii) Diagramas de Sequência e 
(iii) Diagrama de Atividades. 
Os diagramas de Classes e de Componentes não foram incluídos, visto que foram dispensados na etapa.

\subsection{Diagrama de Casos de Uso}

O Diagrama de Casos de Uso apresenta uma visão geral das funcionalidades oferecidas pelo AgroBot ao usuário. O chatbot possui um único ator principal — o \textbf{Agricultor} — que pode desempenhar tanto o papel de comprador quanto de vendedor. Essa decisão simplifica o modelo e reflete a realidade de uso do sistema, no qual qualquer agricultor pode alternar entre publicar anúncios e buscar ofertas conforme sua necessidade.

Os casos de uso foram derivados diretamente dos requisitos funcionais e dos artefatos coletados com agricultores. Dessa forma, garantimos alinhamento entre as necessidades reais do público-alvo e as funcionalidades representadas na modelagem.

Os principais casos de uso identificados são:

\begin{itemize}
    \item \textbf{Cadastrar-se}: o agricultor informa seus dados básicos (nome, município e cultura principal) ao iniciar o uso do bot.
    \item \textbf{Publicar anúncio}: permite criar um anúncio informando categoria, descrição e localização.
    \item \textbf{Listar anúncios}: exibe os anúncios cadastrados no sistema, com paginação automática.
    \item \textbf{Filtrar anúncios}: possibilita refinar a busca por categoria ou região.
    \item \textbf{Visualizar detalhes}: mostra informações completas de um anúncio selecionado.
    \item \textbf{Contatar vendedor}: abre o WhatsApp do anunciante para negociação direta.
    \item \textbf{Gerenciar anúncios}: permite editar, renovar ou excluir anúncios publicados.
    \item \textbf{Receber notificações}: o bot envia alertas com novas ofertas compatíveis com o perfil do agricultor.
\end{itemize}

Alguns relacionamentos foram adicionados para representar dependências entre funcionalidades:

\begin{itemize}
    \item O caso de uso \textbf{Visualizar detalhes} está incluído em \textbf{Listar anúncios}, pois só ocorre após o usuário visualizar uma lista de resultados.
    \item O caso de uso \textbf{Filtrar anúncios} estende \textbf{Listar anúncios}, pois é opcional e complementar.
    \item O caso de uso \textbf{Gerenciar anúncios} estende \textbf{Publicar anúncio}, uma vez que só é possível gerenciar anúncios previamente criados.
\end{itemize}

A Figura a seguir apresenta o Diagrama de Casos de Uso do AgroBot.

\begin{figure}[H]
    \centering
    \includegraphics[width=0.85\textwidth]{../diagramas/caso_de_uso.png}
    \caption{Diagrama de Casos de Uso do AgroBot}
\end{figure}




\postextual
\printbibliography[title={Referências}]


\end{document}
