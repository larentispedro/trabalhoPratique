\documentclass[
  article,
  a4paper,
  12pt,
  fleqn,
  oneside,
  chapter = TITLE,
  section = TITLE,
]{abntex2}

\usepackage[
  BibURLs = false,
  ABNTNum = none,
]{unoesc-article}

\usepackage{caption}

\addbibresource{referencias.bib}

\titulo{AgroBot: Uso de Chatbot no WhatsApp para Conectar Produtores e Compradores de Insumos Agrícolas}

\autor{%
  Pedro Augusto Larentis Maldaner%
  \thanks{%
    \affil{Acadêmico de Sistemas de Informação; UNOESC; Chapecó}%
    \sep\email{pedro.augusto@unoesc.edu.br}%
  }%
  \and Pedro Henrique Tormem%
  \thanks{%
    \affil{Acadêmico de Sistemas de Informação; UNOESC; Chapecó}%
    \sep\email{pedro.tormem@unoesc.edu.br}%
  }%
  \and William Breno Antunes de Lima de Oliveira%
  \thanks{%
    \affil{Acadêmico de Sistemas de Informação; UNOESC; Chapecó}%
    \sep\email{william.o@unoesc.edu.br}%
  }%  
  \and Prof. Jacson Luiz Matte% 
  \thanks{%
    \affil{Especialista em Desenvolvimento de Aplicações Web; UNOPAR; Chapecó}%
    \sep\email{jacson.matte@unoesc.edu.br}%
  }%
}

\data{}

\makeindex
\crefname{figure}{Figura}{Figuras}
\crefname{table}{Quadro}{Quadros}

% Remove símbolos (*, †, ‡, §) dos autores
\renewcommand{\thefootnote}{} 
\makeatletter
\let\@fnsymbol\@arabic
\makeatother

\begin{document}

\pretextual

% ========== CAPA ==========
\begin{paginadetitulo}

 \begin{ambienteresumo}%% Resumo
O AgroBot é um chatbot desenvolvido para WhatsApp com o objetivo de facilitar a comunicação e a negociação entre produtores e compradores de insumos agrícolas. A solução busca suprir a dificuldade enfrentada por agricultores — especialmente pequenos e médios — na divulgação de produtos, troca de informações e acesso a oportunidades de compra e venda. Utilizando uma plataforma já familiar e amplamente utilizada no meio rural, o projeto elimina a necessidade de instalar novos aplicativos e oferece uma interface simples, direta e acessível.

O sistema permitirá o cadastro automático de usuários, publicação e consulta de anúncios de insumos, sementes, maquinários e áreas para arrendamento. Além disso, oferecerá filtragem por categorias, contato direto entre produtores e notificações automáticas de novas ofertas. O AgroBot propõe-se a preencher lacunas existentes em soluções atuais, combinando a praticidade do WhatsApp com funcionalidades típicas de um marketplace agrícola, oferecendo uma experiência inclusiva, moderna e eficiente para o agronegócio.
   
    \end{ambienteresumo}

\end{paginadetitulo}

% ======= FORÇA O TEXTO A COMEÇAR NA SEGUNDA PÁGINA =======
\newpage
\textual

% ========== CONTEÚDO ==========
\section{Introdução}\label{sec:intro}

O setor agrícola enfrenta desafios constantes na aproximação entre produtores, fornecedores de insumos e compradores. Muitas vezes, a comunicação é fragmentada, feita por canais informais e sem organização centralizada, o que dificulta o acesso a sementes, insumos e até mesmo ao aluguel de terras.
Diante da popularização das tecnologias de mensagens instantâneas, como o WhatsApp, surge a oportunidade de utilizar chatbots como ferramenta prática e acessível para facilitar essa interação.
Este projeto propõe o desenvolvimento do AgroBot, um chatbot no WhatsApp que funcione como canal automatizado de comunicação entre agricultores, permitindo que produtores publiquem e acessem anúncios de insumos, sementes, maquinários e até áreas para arrendamento. O sistema oferecerá informações rápidas, organização das demandas e maior visibilidade às oportunidades de negociação, fortalecendo as trocas diretas entre agricultores. Além disso, no futuro poderá contar com um algoritmo de recomendação inteligente, que sugere produtos e serviços de acordo com os interesses do usuário, seu histórico de interações e a sazonalidade das culturas.


\section{Delimitação do Tema}\label{ssec:teor}

O projeto está delimitado ao desenvolvimento de um chatbot para WhatsApp, que servirá como plataforma de intermediação entre agricultores. O foco é facilitar o contato direto entre produtores rurais, eliminando intermediários e permitindo que agricultores anunciem e encontrem produtos, serviços e equipamentos. 

\section{Justificativa}

O uso de aplicativos de mensagens já é consolidado no Brasil, sendo o WhatsApp a principal ferramenta de comunicação entre agricultores e comunidades rurais. Desenvolver um aplicativo próprio exigiria mais tempo e custos, além da necessidade de convencer usuários a instalarem algo novo.
O chatbot, por outro lado, aproveita uma plataforma já difundida, reduzindo barreiras de uso e aumentando as chances de adoção. Essa solução permite democratizar o acesso à informação, reduzir intermediários e agilizar negociações agrícolas, fortalecendo o setor com uma tecnologia simples e de alto impacto.
Além disso, a inclusão de anúncios de insumos, sementes, áreas para arrendamento e maquinários entre agricultores amplia ainda mais o impacto da solução, pois cria uma rede colaborativa de troca e negociação, que favorece especialmente os pequenos e médios produtores. Dessa forma, o AgroBot não apenas aproxima agricultores, mas também promove economia solidária e acesso a recursos que muitas vezes ficam concentrados em grandes mercados.



\section{Objetivo Geral}

Desenvolver um chatbot no WhatsApp para aproximar produtores e compradores agrícolas, facilitando a negociação de insumos, sementes e áreas para arrendamento, de forma prática, acessível e organizada.

Objetivos Específicos:

\begin{itemize}
  \item Implementar um sistema de cadastro automatizado de produtores e compradores;
  \item Disponibilizar listagem e divulgação de produtos, serviços, áreas agrícolas e maquinários;
  \item Criar uma busca inteligente por categorias;
  \item Proporcionar um canal de comunicação ágil e direto entre agricultores;
  \item Desenvolver um sistema de recomendação personalizada, que sugira produtos e serviços de acordo com o perfil do usuário, seu histórico de interações e a sazonalidade agrícola.
\end{itemize}

\section{Trabalhos Relacionados}

Durante o desenvolvimento do projeto AgroBot, foram analisados sistemas e soluções tecnológicas voltadas ao setor agropecuário que utilizam chatbots, plataformas digitais e automações para comunicação e negociação entre produtores. O objetivo desta análise é identificar boas práticas, limitações e oportunidades de diferenciação para o projeto.

\subsection{Nestlé — Theo}

A Nestlé desenvolveu o Theo, um assistente virtual criado para auxiliar produtores de cacau. O chatbot fornece informações técnicas sobre boas práticas agrícolas e manejo sustentável via WhatsApp, utilizando linguagem simples e adaptada à realidade do campo. De acordo com a Nestlé \cite{nestle_theo_mpm, nestle_theo_folha}, o objetivo é oferecer suporte técnico automatizado e acessível aos agricultores.

\textbf{Pontos fortes:} fornece informações técnicas padronizadas e linguagem próxima ao agricultor.  
\textbf{Limitações:} restrito a uma única cultura (cacau) e não atua como plataforma de anúncios ou negociação.  
\textbf{Diferencial do AgroBot:} o foco está na intermediação comercial entre produtores, com recomendações baseadas em histórico e sazonalidade das culturas.

\subsection{Agrofy}

O Agrofy é uma plataforma online voltada à compra e venda de produtos e serviços agrícolas, permitindo anúncios de tratores, sementes, insumos e maquinários.  
Segundo informações disponíveis no site oficial \cite{agrofy}, a plataforma funciona como um marketplace do agronegócio com grande variedade de produtos e interface moderna.

\textbf{Pontos fortes:} ampla base de produtos e interface profissional.  
\textbf{Limitações:} requer cadastro e uso via navegador, o que pode afastar pequenos produtores que utilizam apenas o WhatsApp.  
\textbf{Diferencial do AgroBot:} acesso direto via WhatsApp, sem necessidade de instalação adicional e com interface conversacional simplificada.

\subsection{Agritek Bot}

O Agritek Bot, desenvolvido pela empresa brasileira Agritek \cite{agritek}, é um chatbot que automatiza o atendimento de produtores via WhatsApp. Ele permite o envio de mensagens automáticas, catálogos de produtos e suporte técnico, otimizando a comunicação entre fornecedores e agricultores.

\textbf{Pontos fortes:} automatiza o atendimento via WhatsApp e facilita a comunicação entre empresas e produtores.  
\textbf{Limitações:} não realiza intermediação entre produtores e compradores; seu foco é o atendimento comercial.  
\textbf{Diferencial do AgroBot:} possibilita a criação de um espaço de anúncios e trocas diretas entre agricultores, favorecendo negociações entre pequenos produtores.

\subsection{WhatsWave Agro}

O WhatsWave Agro é uma plataforma que permite criar agentes de inteligência artificial e automações para o agronegócio dentro do WhatsApp \cite{whatswave}. É utilizada por fazendas e empresas rurais para enviar alertas, gerenciar tarefas e automatizar comunicações internas.

\textbf{Pontos fortes:} bom uso de automações e comunicação via WhatsApp.  
\textbf{Limitações:} não atua como marketplace nem promove a intermediação comercial entre produtores.  
\textbf{Diferencial do AgroBot:} combina o uso de chatbot e marketplace em um ambiente conversacional acessível e de baixo custo.

\subsection{Análise Comparativa e Diferencial do AgroBot}

A análise dos sistemas estudados mostra que, embora existam diversas soluções digitais no setor agrícola, poucas exploram o potencial dos \textit{chatbots} para negociações diretas entre produtores. O AgroBot se diferencia por unir a praticidade do atendimento automatizado via WhatsApp com funcionalidades de um marketplace, promovendo um espaço acessível, intuitivo e inclusivo para anúncios e trocas agrícolas.


\section{Requisitos Funcionais}
\begin{itemize}

  \item \textbf{RF01 – Cadastro via conversa:} o bot coleta nome, município/estado e cultura principal durante uma conversa guiada.
  \item \textbf{RF02 – Publicar anúncio:} fluxo guiado onde o bot solicita categoria, descrição e local do anúncio.
  \item \textbf{RF03 – Listar anúncios:} permite ao usuário buscar por termos ou categorias e visualizar resultados paginados.
  \item \textbf{RF04 – Filtro simples por categoria e local:} busca refinada por tipo e região.
  \item \textbf{RF05 – Contato entre comprador e vendedor:} exibe o número do anunciante ou inicia mensagem automática.
  \item \textbf{RF06 – Gerenciar anúncios:} comandos para editar, renovar ou excluir.
  \item \textbf{RF07 – Notificações proativas:} envio de mensagens com novas ofertas regionais.
  \item \textbf{RF08 – Relatórios básicos (admin):} painel de estatísticas e categorias mais populares.
\end{itemize}


\section{Requisitos Não Funcionais}
\begin{itemize}
  \item \textbf{RNF01 – Interface simples e intuitiva:} linguagem acessível e menus guiados, adequados a produtores com pouca experiência tecnológica.
  \item \textbf{RNF02 – Compatibilidade com dispositivos móveis:} integração via WhatsApp Business API, sem necessidade de instalar outros aplicativos.
  \item \textbf{RNF03 – Tempo de resposta e desempenho:} respostas automáticas com média inferior a 3 segundos.
  \item \textbf{RNF04 – Segurança e privacidade:} proteção dos dados pessoais e uso ético das informações.
  \item \textbf{RNF05 – Disponibilidade e manutenção:} arquitetura leve e de fácil manutenção.
  \item \textbf{RNF06 – Disponibilidade:} Código aberto e acessível no GitHub
\end{itemize}

\postextual
\printbibliography[title={Referências}]


\end{document}
